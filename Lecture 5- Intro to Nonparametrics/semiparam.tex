\documentclass[11pt,handout,xcolor=pdftex,dvipsnames,table,mathserif,aspectratio=169]{beamer}
\usetheme{metropolis}
%\usetheme{Darmstadt}
%\usepackage{times}
%\usefonttheme{structurebold}

\usepackage[english]{babel}
%\usepackage[table]{xcolor}
\usepackage{amsmath}
\usepackage{pgf,pgfarrows,pgfnodes,pgfautomata,pgfheaps}
\usepackage{amsmath,amssymb,setspace}
\usepackage[latin1]{inputenc}
\usepackage[T1]{fontenc}
\usepackage{relsize}

%\usepackage{pxfonts}

\usepackage[absolute,overlay]{textpos} 
\newenvironment{reference}[2]{% 
  \begin{textblock*}{\textwidth}(#1,#2) 
      \footnotesize\it\bgroup\color{red!50!black}}{\egroup\end{textblock*}} 

\DeclareMathSizes{10}{10}{6}{6} 


\title [Nonparametrics]{Nonparametrics and Local Methods: Semiparametrics}
\author{C.Conlon}
\institute{Applied Econometrics}
\date{\today}
\setbeamerfont{equation}{size=\tiny}
\begin{document}

\begin{frame}
\titlepage
\end{frame}



 \frame{\frametitle{The
Seminonparametric Approach}
\begin{itemize}
\item If we are ``pretty sure'' that $f$ is almost $f_{m,\sigma}$ for some
family of densities indexed by  $(m,\sigma)$, then we can choose a family of positive functions of increasing complexity  $P^1_\theta,P^2_\theta,\ldots$
\item Choose some $M$ that goes to infinity as $n$ does (more slowly), and
maximize over $(m,\sigma,\theta)$ the loglikelihood

\[
\sum_{i=1}^n \log f_{m,\sigma}(y_i)P^M_\theta(y_i).
\]
It works\ldots but it is hard to constrain it to be a density for
large $M$. \end{itemize}}



\frame{\frametitle{Mixtures of Normals}

\pause

A special case of seminonparametrics, and
usually a very good approach: Let $y|x$ be drawn from

\pause

\begin{description}
\item[] $N(m_1(x,\theta),\sigma^2_1(x,\theta))$ with probability
$q_1(x,\theta)$;
\item[] \ldots
\item[] $N(m_K(x,\theta),\sigma^2_K(x,\theta))$ with probability
$q_K(x,\theta).$
\end{description}
where you choose some parameterizations, and the $q_k$'s are positive
and sum to 1.

\pause

Can be estimated by maximum-likelihood:


\[
\max_{\theta} \sum_{i=1}^n  \log\left(
\sum_{k=1}^K
\frac{q_k(x_i,\theta)}{\sigma_k(x,\theta)}
\phi\left(\frac{y_i-m_k(x_i,\theta)}{\sigma_k(x_i,\theta)}\right)\right).
\]


\pause

Usually works very well with $K\leq 3$ (perhaps after transforming
$y$ to $\log y$, e.g).


}



 
\frame{\frametitle{Splines: trading off fit and smoothness}

\pause

Choose some $0<\lambda<\infty$ and
\[
\min_{m(.)} \sum_i (y_i-m(x_i))^2+\lambda J(m),
\]

\pause

Then we ``obtain'' the
natural cubic spline with knots=$(x_1,\ldots,x_n)$:

\pause

\begin{itemize}[<+->]
\item $m$ is a cubic polynomial between consecutive $x_i$'s
\item it is linear out-of-sample
\item it is $C^2$ everywhere. 
\end{itemize}

\pause

``Consecutive'' implies one-dimensional\ldots harder to generalize to
$p_x>1$.

\pause

\textbf{Orthogonal polynomials:} check out Chebyshev, $1, x,
2x^2-1,4x^3-3x\ldots$ (on $[-1,1]$ here.)
}


\begin{frame}
\frametitle{Additive models}
\pause
{\em Additive model:} $y=\alpha+\sum_{j=1}^p + f_j (X_j) + +\epsilon$\\
\pause
{\em Backfitting algorithm:} start with $\hat{a}=\overline{y}_n$, and some zero--mean
guesses  $\hat{f}_j \equiv 0 $.
\pause
Then for $j=1,\ldots,p,\ldots ,1,2,\ldots,p,\ldots$,
\pause
\begin{enumerate}[<+->]
\item   Define 
\begin{eqnarray*}
f_j &\leftarrow& S_j[\{y_i - \hat{\alpha} - \sum_{k\neq j} \hat{f}_k (x_{ik})\}_1^N ]\\
f_j &\leftarrow&  \hat{f}_j - \frac{1}{N} \sum_{i=1}^N \hat{f}_j (x_{ij}).
\end{eqnarray*}
\item Regress $\hat{y}$ on $x_j$ to get $R_j$; then replace $\hat{r}_j$ with $R_j-\frac{1}{n}\sum_i \hat{r}_j(x_{ji})$ (where $S_j$ is some cubic smoothing spline).
\item Iterate until $\hat{f}_j$ doesn't change.
\end{enumerate}
\end{frame}



\end{document}

